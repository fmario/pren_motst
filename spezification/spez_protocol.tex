\documentclass[10pt,a4paper]{article}
\usepackage[utf8]{inputenc}
\usepackage[left=3.50cm, right=3.00cm, top=3.00cm, bottom=3.00cm]{geometry}
\usepackage{amsmath}
\usepackage{amsfonts}
\usepackage{amssymb}
\usepackage{graphicx}
\usepackage{multirow}
\usepackage{pdflscape}
\usepackage{tabularx}
\usepackage{array}

\title{Specification for Communication Protocol}
\author{}
\date{}

    \renewcommand{\arraystretch}{1.7}
    
\newcommand{\BL}{9 }
% Befehlscodes:
%--------------------------------------------------------
\newcommand{\IMV}{0x00}
\newcommand{\MTO}{0x01}
\newcommand{\WTM}{0x02}
\newcommand{\IRD}{0x03}
\newcommand{\MOV}{0x04}
\newcommand{\SPM}{0x05}
\newcommand{\GAP}{0x06}
\newcommand{\STP}{0x07}
\newcommand{\GTP}{0x08}
\newcommand{\CNP}{0x09}
\newcommand{\DCM}{0x0E}

\newcommand{\SVH}{0x0A}
\newcommand{\GOH}{0x0B}
\newcommand{\SWP}{0x0C}
\newcommand{\MWP}{0x0D}
%--------------------------------------------------------

\begin{document}

\maketitle
\tableofcontents
\section*{Definitions}
FALSE := 0x00\\
TRUE := 0x01..0xFF\\

\section{Packet Structur}
\subsection{Command Structur}
\begin{tabular}{|p{2.2cm}|p{3.2cm}|p{2cm}|}
	\hline 
	Payload \mbox{(0..\BL Bytes)} & 
	Padding \mbox{(\BL Bytes - Length)} & 
	\textit{Checksum \mbox{(1 Byte)}} \\
	\hline 
\end{tabular}  
\\\\
\begin{tabbing}
	\hspace*{2cm} \= \kill
	Payload: \>Befehlsfolge\\
	Padding: \>Füllbytes\\
	\textit{Checksum}: \>\textit{Prüfsumme CRC8(Payload) (optional)}\\
\end{tabbing}

\subsection{Response Structur} 
\begin{tabular}{|p{1.5cm}|p{2.2cm}|p{3.2cm}|p{2cm}|}
	\hline 
	Ack \mbox{(1 Byte)} &
	Payload \mbox{(0..3 Bytes)} & 
	Padding \mbox{(3 Bytes - Length)} & 
	\textit{Checksum \mbox{(1 Byte)}} \\
	\hline 
\end{tabular}  
\\\\
\begin{tabbing}
	\hspace*{2cm} \= \kill
	Ack:	 \>Acknowledge (TRUE wenn Befehl gültig)\\
	Payload: \>Antwort\\
	Padding: \>Füllbytes\\
	\textit{Checksum}: \>\textit{Prüfsumme CRC8(Payload) (optional)}\\
\end{tabbing}


\section{Command List}
\subsection{Generic Commands}
\begin{tabular}{|p{1.8cm}|p{1cm}|p{1cm}|p{1.3cm}|p{1.2cm}|p{1cm}|p{1cm}|p{1cm}|p{1cm}|p{1cm}|}
	\hline
	\multirow{2}{2cm}{Command mnemonic} & \multicolumn{9}{c|}{Command Bytes (Payload)} \\\cline{2-10} 
		&  Byte 0 & Byte 1 & Byte 2 & Byte 3 & Byte 4 & Byte 5 & Byte 6 & Byte 7 & Byte 8\\ \hline
	InitMove	& \IMV	& Motor	& Dir & Speed & Acc & Dec & & & \\\hline
	MoveTo		& \MTO	& Motor	& Dir & AbsPos (HB) & AbsPos & AbsPos (LB) & Speed & Acc & Dec \\\hline
	WaitMoved	& \WTM	& Motor	& Timeout (HB) & Timeout (LB)& & & & & \\\hline
	IsReady	& \IRD	& Motor	& & & & & & & \\\hline
	Move	& \MOV	& Motor	& Dir & Speed & Acc & Dec & & & \\\hline
	StopMove & \SPM& Motor	& IsHard Stop & & & & & & \\\hline
	GetAbsPos& \GAP& Motor	& & & & & & & \\\hline
	SetPin	& \STP	& PinNr	& IsHigh & & & & & & \\\hline
	GetPin	& \GTP	& PinNr	& & & & & & & \\\hline
	ConfigPin& \CNP& PinNr	& IsOutput & & & & & & \\\hline	
	DcMove	& \DCM & Dir & Time (HB) & Time (LB) & goHiz & & & & \\\hline
\end{tabular}
\\\\

\subsection{Specific Commands}
\begin{tabular}{|p{1.8cm}|p{1cm}|p{1cm}|p{1.4cm}|p{1cm}|p{1cm}|p{1cm}|p{1cm}|p{1cm}|p{1cm}|}
	\hline
	\multirow{2}{2cm}{Command mnemonic} & \multicolumn{9}{c|}{Command Bytes (Payload)} \\\cline{2-10} 
		&  Byte 0 & Byte 1 & Byte 2 & Byte 3 & Byte 4 & Byte 5 & Byte 6 & Byte 7 & Byte 8 \\ \hline
	SaveHome	& \SVH	& Motor	& & & & & & & \\\hline
	GoHome		& \GOH	& Motor	& & & & & & & \\\hline
	Save \mbox{WayPoint}& \SWP	& Motor	& & & & & & & \\\hline
	MoveTo WayPoint& \MWP& Motor	& WayPoint & Speed & Acc & Dec & & & \\\hline
\end{tabular}


\subsection{InitMove (MOTOR, DIR)}
Command code: \textbf{\IMV}\\\\
\newcolumntype{C}{>{\centering\arraybackslash}p{0.8cm}}
\begin{tabular}{|p{1.2cm}|C|C|C|C|C|C|C|C|}
	\hline
 		& Bit 7 & Bit 6 & Bit 5 & Bit 4 & Bit 3 & Bit 2 & Bit 1 & Bit 0 \\\hline
	Byte 0 &  &  &  &  &  &  &  &  \\ \hline
	Byte 1 &  \multicolumn{8}{c|}{MOTOR}  \\ \hline
	Byte 2 &  \multicolumn{8}{c|}{DIR}  \\ \hline
	Byte 3 &  \multicolumn{8}{c|}{SPEED}  \\ \hline
	Byte 4 &  \multicolumn{8}{c|}{ACC}  \\ \hline
	Byte 5 &  \multicolumn{8}{c|}{DEC}  \\ \hline
\end{tabular}
\paragraph*{Description\\}
Der gewählte MOTOR fährt zur Initialisierung in der angegebenen Richtung (DIR) bis an eine Endposition (zB. mechanischer Anschlag).
Sobald diese Position erreicht ist, bleibt der Motor stehen und ist bereit.\\ 

\subsection{MoveTo (MOTOR, DIR, ABS\_POS, SPEED, ACC, DEC)}
Command code: \textbf{\MTO}\\\\
\newcolumntype{C}{>{\centering\arraybackslash}p{0.8cm}}
\begin{tabular}{|p{1.2cm}|C|C|C|C|C|C|C|C|}
	\hline
 		& Bit 7 & Bit 6 & Bit 5 & Bit 4 & Bit 3 & Bit 2 & Bit 1 & Bit 0 \\\hline
	Byte 0 &  &  &  &  &  &  &  &  \\ \hline
	Byte 1 &  \multicolumn{8}{c|}{MOTOR}  \\ \hline
	Byte 2 &  \multicolumn{8}{c|}{DIR}  \\ \hline
	Byte 3 &  \multicolumn{8}{c|}{ABS\_POS (High Byte)}  \\ \hline
	Byte 4 &  \multicolumn{8}{c|}{ABS\_POS (Middle Byte)}  \\ \hline
	Byte 5 &  \multicolumn{8}{c|}{ABS\_POS (Low Byte)}  \\ \hline
	Byte 6 &  \multicolumn{8}{c|}{SPEED}  \\ \hline
	Byte 7 &  \multicolumn{8}{c|}{ACC}  \\ \hline
	Byte 8 &  \multicolumn{8}{c|}{DEC}  \\ \hline
\end{tabular}
\paragraph*{Description\\}
Der gewählt MOTOR fährt zu eine bestimmten absoluten Position (ABS\_POS). Dabei kann dem Motor eine Richtung (DIR) sowie eine Geschwindigkeit (SPEED in $\frac{Steps}{s}$) vorgegeben werden. Des weiteren kann eine Beschleunigungsrampe (ACC in $\frac{Steps}{s^2}$) und eine Verzögerungsrampe vorgegeben werden (DEC in $\frac{Steps}{s^2}$).\\
Falls Standardwerte für Geschwindigkeit, Beschleunigung und Verzögerung verwendet werden sollen, können die jeweiligen Parameter auf 0x00 gesetzt werden.\\

\subsection{WaitMoved (MOTOR, TIMEOUT)}
Command code: \textbf{\WTM}\\\\
\newcolumntype{C}{>{\centering\arraybackslash}p{0.8cm}}
\begin{tabular}{|p{1.2cm}|C|C|C|C|C|C|C|C|}
	\hline
 		& Bit 7 & Bit 6 & Bit 5 & Bit 4 & Bit 3 & Bit 2 & Bit 1 & Bit 0 \\\hline
	Byte 0 &  &  &  &  &  &  &  &  \\ \hline
	Byte 1 &  \multicolumn{8}{c|}{MOTOR}  \\ \hline
	Byte 2 &  \multicolumn{8}{c|}{TIMEOUT (High Byte)}  \\ \hline
	Byte 3 &  \multicolumn{8}{c|}{TIMEOUT (Low Byte)}  \\ \hline
\end{tabular}
\paragraph*{Description\\}
Bei diesem Befehl wird die Antwort erst zurück gesendet, wenn der spezifizierte MOTOR still steht. Zusätzlich muss ein TIMEOUT (in $ms$) mitgegeben werden, nach welchem die Antwort auf jeden Fall gesendet wird. Falls das Timeout abgelaufen ist, wird ein Errorcode mitgesendet.\\

\subsection{IsReady (MOTOR)}
Command code: \textbf{\IRD}\\\\
\newcolumntype{C}{>{\centering\arraybackslash}p{0.8cm}}
\begin{tabular}{|p{1.2cm}|C|C|C|C|C|C|C|C|}
	\hline
 		& Bit 7 & Bit 6 & Bit 5 & Bit 4 & Bit 3 & Bit 2 & Bit 1 & Bit 0 \\\hline
	Byte 0 &  &  &  &  &  &  &  &  \\ \hline
	Byte 1 &  \multicolumn{8}{c|}{MOTOR}  \\ \hline
\end{tabular}
\paragraph*{Description\\}
Sendet mit der Antwort, ob der spezifizierte MOTOR bereit ist. Die Antwort ist TRUE, wenn der Motor keinen aktuellen Befehl am ausführen ist, andernfalls ist sie FALSE.\\

\subsection{Move (MOTOR, DIR, SPEED, ACC, DEC)}
Command code: \textbf{\MOV}\\\\
\newcolumntype{C}{>{\centering\arraybackslash}p{0.8cm}}
\begin{tabular}{|p{1.2cm}|C|C|C|C|C|C|C|C|}
	\hline
 		& Bit 7 & Bit 6 & Bit 5 & Bit 4 & Bit 3 & Bit 2 & Bit 1 & Bit 0 \\\hline
	Byte 0 &  &  &  &  &  &  &  &  \\ \hline
	Byte 1 &  \multicolumn{8}{c|}{MOTOR}  \\ \hline
	Byte 2 &  \multicolumn{8}{c|}{DIR}  \\ \hline
	Byte 3 &  \multicolumn{8}{c|}{SPEED}  \\ \hline
	Byte 4 &  \multicolumn{8}{c|}{ACC}  \\ \hline
	Byte 5 &  \multicolumn{8}{c|}{DEC}  \\ \hline
\end{tabular}
\paragraph*{Description\\}
Der gewählt MOTOR fährt mit der angegebenen Geschwindigkeit (SPEED in $\frac{Steps}{s}$) in die Richtung DIR. Es kann eine Beschleunigungsrampe (ACC in $\frac{Steps}{s^2}$) und eine Verzögerungsrampe vorgegeben werden (DEC in $\frac{Steps}{s^2}$).\\
Falls Standardwerte für Geschwindigkeit, Beschleunigung und Verzögerung verwendet werden sollen, können die jeweiligen Parameter auf 0x00 gesetzt werden.\\

\subsection{StopMove (MOTOR, IS\_HARDSTOP)}
Command code: \textbf{\SPM}\\\\
\newcolumntype{C}{>{\centering\arraybackslash}p{0.8cm}}
\begin{tabular}{|p{1.2cm}|C|C|C|C|C|C|C|C|}
	\hline
 		& Bit 7 & Bit 6 & Bit 5 & Bit 4 & Bit 3 & Bit 2 & Bit 1 & Bit 0 \\\hline
	Byte 0 &  &  &  &  &  &  &  &  \\ \hline
	Byte 1 &  \multicolumn{8}{c|}{MOTOR}  \\ \hline
	Byte 2 &  \multicolumn{8}{c|}{IS\_HARDSTOP}  \\ \hline
\end{tabular}
\paragraph*{Description\\}
Stoppt den MOTOR. Mit dem zusätzlichen Parameter IS\_HARDSTOP kann angegeben werden, ob der Motor auf der Stelle stoppt (IS\_HARDSTOP = TRUE) oder mit der programmierten Verzögerungsrampe abbremst (IS\_HARDSTOP = FALSE).\\

\subsection{GetAbsPos (MOTOR)}
Command code: \textbf{\GAP}\\\\
\newcolumntype{C}{>{\centering\arraybackslash}p{0.8cm}}
\begin{tabular}{|p{1.2cm}|C|C|C|C|C|C|C|C|}
	\hline
 		& Bit 7 & Bit 6 & Bit 5 & Bit 4 & Bit 3 & Bit 2 & Bit 1 & Bit 0 \\\hline
	Byte 0 &  &  &  &  &  &  &  &  \\ \hline
	Byte 1 &  \multicolumn{8}{c|}{MOTOR}  \\ \hline
\end{tabular}
\paragraph*{Description\\}
Fragt die aktuelle absolute Position des MOTORs ab. Die absolute Position wird in der Antwort mit 3 Bytes beschrieben, wobei das höchste Byte als erstes geschickt wird.\\

\subsection{SetPin (PIN\_NR,IS\_HIGH)}
Command code: \textbf{\STP}\\\\
\newcolumntype{C}{>{\centering\arraybackslash}p{0.8cm}}
\begin{tabular}{|p{1.2cm}|C|C|C|C|C|C|C|C|}
	\hline
 		& Bit 7 & Bit 6 & Bit 5 & Bit 4 & Bit 3 & Bit 2 & Bit 1 & Bit 0 \\\hline
	Byte 0 &  &  &  &  &  &  &  &  \\ \hline
	Byte 1 &  \multicolumn{8}{c|}{PIN\_NR}  \\ \hline
	Byte 1 &  \multicolumn{8}{c|}{IS\_HIGH}  \\ \hline
\end{tabular}
\paragraph*{Description\\}
Dieser Befehlt setzt einen IO-Pin (PIN\_NR) des $\mu$C auf den entsprechenden Pegel (IS\_HIGH). Der Pin wird auf Vcc gesetzt, wenn der Parameter IS\_HIGH = TRUE ist, andernfalls auf 0V.\\
Der IO-Pin muss als Output definiert sein. Wenn der gewählte Pin nicht gesetzt werden kann, enthält die Antwort einen Errorcode.\\

\subsection{GetPin (PIN\_NR)}
Command code: \textbf{\GTP}\\\\
\newcolumntype{C}{>{\centering\arraybackslash}p{0.8cm}}
\begin{tabular}{|p{1.2cm}|C|C|C|C|C|C|C|C|}
	\hline
 		& Bit 7 & Bit 6 & Bit 5 & Bit 4 & Bit 3 & Bit 2 & Bit 1 & Bit 0 \\\hline
	Byte 0 &  &  &  &  &  &  &  &  \\ \hline
	Byte 1 &  \multicolumn{8}{c|}{PIN\_NR}  \\ \hline
\end{tabular}
\paragraph*{Description\\}
Mit diesem Befehl kann der Wert eines IO-Pins (PIN\_NR) des $\mu$C abgefragt werden. In der Antwort ist der Zustand des Pins direkt enthalten (Low = FALSE, High = 0x01).\\
Wenn der gewählte Pin nicht gelesen werden kann, enthält die Antwort einen Errorcode.\\

\subsection{ConfigPin (PIN\_NR, IS\_OUTPUT)}
Command code: \textbf{\CNP}\\\\
\newcolumntype{C}{>{\centering\arraybackslash}p{0.8cm}}
\begin{tabular}{|p{1.2cm}|C|C|C|C|C|C|C|C|}
	\hline
 		& Bit 7 & Bit 6 & Bit 5 & Bit 4 & Bit 3 & Bit 2 & Bit 1 & Bit 0 \\\hline
	Byte 0 &  &  &  &  &  &  &  &  \\ \hline
	Byte 1 &  \multicolumn{8}{c|}{PIN\_NR}  \\ \hline
	Byte 2 &  \multicolumn{8}{c|}{IS\_OUTPUT}  \\ \hline
\end{tabular}
\paragraph*{Description\\}
Der gewählte IO-Pin (PIN\_NR) des $\mu$Cs wird anhand des zweiten Parameters (IS\_OUTPUT) konfiguriert. Der Pin wird zum Output, wenn IS\_OUPUT = TRUE ist, andernfalls zum Input.\\

\subsection{DcMove (DIR, TIME, GO\_HiZ)}
Command code: \textbf{\DCM}\\\\
\newcolumntype{C}{>{\centering\arraybackslash}p{0.8cm}}
\begin{tabular}{|p{1.2cm}|C|C|C|C|C|C|C|C|}
	\hline
 		& Bit 7 & Bit 6 & Bit 5 & Bit 4 & Bit 3 & Bit 2 & Bit 1 & Bit 0 \\\hline
	Byte 0 &  &  &  &  &  &  &  &  \\ \hline
	Byte 1 &  \multicolumn{8}{c|}{DIR}  \\ \hline
	Byte 2 &  \multicolumn{8}{c|}{TIME (HB)}  \\ \hline
	Byte 3 &  \multicolumn{8}{c|}{TIME (LB)}  \\ \hline
	Byte 4 &  \multicolumn{8}{c|}{GO\_HiZ}  \\ \hline
\end{tabular}
\paragraph*{Description\\}
Der DC-Motor fährt die angegebene Zeit (TIME in $ms$) in die Richtung DIR.
Wenn GO\_HiZ auf TRUE ist, wird der Motor nach dem Fahren so geschaltet,
dass dieser ein Haltemoment hat.\\

\subsection{SaveHome (MOTOR)}
Command code: \textbf{\SVH}\\\\
\newcolumntype{C}{>{\centering\arraybackslash}p{0.8cm}}
\begin{tabular}{|p{1.2cm}|C|C|C|C|C|C|C|C|}
	\hline
 		& Bit 7 & Bit 6 & Bit 5 & Bit 4 & Bit 3 & Bit 2 & Bit 1 & Bit 0 \\\hline
	Byte 0 &  &  &  &  &  &  &  &  \\ \hline
	Byte 1 &  \multicolumn{8}{c|}{MOTOR}  \\ \hline
\end{tabular}
\paragraph*{Description\\}
Speichert die aktuelle Position als Home-Position für den gewählten MOTOR.\\

\subsection{GoHome (MOTOR)}
Command code: \textbf{\GOH}\\\\
\newcolumntype{C}{>{\centering\arraybackslash}p{0.8cm}}
\begin{tabular}{|p{1.2cm}|C|C|C|C|C|C|C|C|}
	\hline
 		& Bit 7 & Bit 6 & Bit 5 & Bit 4 & Bit 3 & Bit 2 & Bit 1 & Bit 0 \\\hline
	Byte 0 &  &  &  &  &  &  &  &  \\ \hline
	Byte 1 &  \multicolumn{8}{c|}{MOTOR}  \\ \hline
\end{tabular}
\paragraph*{Description\\}
Der entsprechende MOTOR fährt auf die Home-Position\\

\subsection{SaveWayPoint (MOTOR)}
Command code: \textbf{\SWP}\\\\
\newcolumntype{C}{>{\centering\arraybackslash}p{0.8cm}}
\begin{tabular}{|p{1.2cm}|C|C|C|C|C|C|C|C|}
	\hline
 		& Bit 7 & Bit 6 & Bit 5 & Bit 4 & Bit 3 & Bit 2 & Bit 1 & Bit 0 \\\hline
	Byte 0 &  &  &  &  &  &  &  &  \\ \hline
	Byte 1 &  \multicolumn{8}{c|}{MOTOR}  \\ \hline
\end{tabular}
\paragraph*{Description\\}
Speichert die aktuelle Position als Wegpunkt für den gewählten MOTOR. In der Antwort ist die Nummer des gespeicherten Wegpunktes enthalten (max. 255).\\

\subsection{MoveToWayPoint (MOTOR, WAY\_POINT, SPEED, ACC, DEC)}
Command code: \textbf{\MWP}\\\\
\newcolumntype{C}{>{\centering\arraybackslash}p{0.8cm}}
\begin{tabular}{|p{1.2cm}|C|C|C|C|C|C|C|C|}
	\hline
 		& Bit 7 & Bit 6 & Bit 5 & Bit 4 & Bit 3 & Bit 2 & Bit 1 & Bit 0 \\\hline
	Byte 0 &  &  &  &  &  &  &  &  \\ \hline
	Byte 1 &  \multicolumn{8}{c|}{MOTOR}  \\ \hline
	Byte 2 &  \multicolumn{8}{c|}{WAY\_POINT}  \\ \hline
	Byte 3 &  \multicolumn{8}{c|}{SPEED}  \\ \hline
	Byte 4 &  \multicolumn{8}{c|}{ACC}  \\ \hline
	Byte 5 &  \multicolumn{8}{c|}{DEC}  \\ \hline
\end{tabular}
\paragraph*{Description\\}
Der MOTOR fährt zum angegebenen Wegpunkt (WAY\_POINT). Dabei kann die Geschwindigkeit ((SPEED in $\frac{Steps}{s}$)), sowie eine Beschleunigungsrampe (ACC in $\frac{Steps}{s^2}$) und Verzögerungsrampe (DEC in $\frac{Steps}{s^2}$) vorgegeben werden.\\
Falls Standardwerte für Geschwindigkeit, Beschleunigung und Verzögerung verwendet werden sollen, können die jeweiligen Parameter auf 0x00 gesetzt werden.\\

\newpage
\section{Errorcodes}
Errorcodes werden mit der Response Structure gesendet. 
Das Acknowledge Byte ist auf den Wert 0x00 und das erste Byte beinhaltet den Error-Code.\\\\
\begin{tabular}{|p{1.2cm}|p{2cm}|p{4cm}|p{2cm}|}
	\hline
 		Ack & Byte 0 & Byte 1 - Byte 2 & \textit{Checksum}\\\hline
	 	0x00 & Error code & Padding (0x00) & \\ \hline
\end{tabular}\\\\

\subsection{Full Buffer}
Error code: \textbf{0xE0}
\paragraph*{Description\\}
Der Prozessor hat noch nicht alle Befehle abgearbeitet. Der Befehls-Buffer ist voll und es können keine Weiteren Befehle gespeichert werden.

\subsection{Invalide Command}

Error code: \textbf{0xE1}
\paragraph*{Description\\}
Es wurde ein ungültiger Befehl geschickt, welche nicht verarbeitet werden kann.

\subsection{Invalide Address}

Error code: \textbf{0xE2}
\paragraph*{Description\\}
Die Motorennummer ist ungültig.

\subsection{Motor not ready}

Error code: \textbf{0xE3}
\paragraph*{Description\\}
Der Motor ist noch nicht bereit (Befindet sich noch in Bewegung).

\subsection{Motor Error}

Error code: \textbf{0xE4}
\paragraph*{Description\\}
Der Motor hat einen Error.

\subsection{Full Way Point Buffer}

Error code: \textbf{0xE5}
\paragraph*{Description\\}
Die maximale Anzahl Way Points für diesen Motor sind erreicht.

\subsection{Invalid Way Point}

Error code: \textbf{0xE6}
\paragraph*{Description\\}
Der angegebene Way Point existiert nicht.
\end{document}

\subsection{Wrong Pin Number}

Error code: \textbf{0xE7}
\paragraph*{Description\\}
Der angegebene Pin kann nicht bearbeitet werden.
\end{document}